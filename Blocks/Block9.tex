\chapter*{Block 9: The Labour Market}

\section*{Factors of Production}
The main factors of production are:
\begin{itemize}
    \item \textbf{Land:} All free gifts of nature (land, forests, minerals). Supply is generally fixed, even in the long run.
    \item \textbf{Labour:} Mental and physical effort of people employed for remuneration. Individual skills and qualifications constitute human capital.
    \item \textbf{Physical Capital:} Stock of produced goods used in production. It can be increased and depreciates over time.
\end{itemize}

Other factors sometimes included are raw materials (fully consumed in production) and entrepreneurship (innovation and risk-taking).

\section*{Firm's Demand for Inputs}
\subsection*{Long Run}
In the long run, firms can vary all inputs. An increase in the price of labor (wage rate) decreases labor demand due to:
\begin{itemize}
    \item \textbf{Substitution effect:} Capital becomes relatively cheaper.
    \item \textbf{Output effect:} Higher costs reduce the profit-maximizing output level.
\end{itemize}

The elasticity of demand for the firm's output affects the output effect; less elastic demand implies a smaller output effect of a wage increase.

\subsection*{Short Run}
In the short run, at least one factor is fixed. Labor is subject to diminishing marginal returns when other factors are fixed. The firm hires labor until:
\[
\text{Marginal Cost of Labor (MCL)} = \text{Marginal Revenue Product of Labor (MRPL)}
\]
For a price-taking firm:
\[
\text{MRPL} = \text{Price of Output} \times \text{Marginal Product of Labor (MPL)}
\]
For a firm with market power:
\[
\text{MRPL} = \text{Marginal Revenue (MR)} \times \text{MPL}
\]
The optimal rule for hiring labor is to hire until:
\[
W = \text{MRPL} = \text{MR} \times \text{MPL}
\]
For a competitive firm where price = marginal revenue (\(MR = P\)):
\[
W = P \times \text{MPL}
\]
For a firm with market power (e.g., a monopolist):
\[
W = P \left(1 + \frac{1}{\epsilon}\right) \times \text{MPL}
\]
where \(\epsilon\) is the price elasticity of demand for output. Note that because \(\epsilon\) is negative, the labor demand curve for a non-competitive firm lies below the labor demand curve for a competitive firm with the same MPL curve and is steeper.

\section*{Industry Demand for Labor}
The industry demand curve for labor is steeper than the horizontal sum of individual firms' short-run demand curves. This is because lower wages increase industry output, leading to a fall in the output price and shifting the firms' curves.

\section*{Labor Supply Decisions}
Labor supply depends on:
\begin{itemize}
    \item Population size
    \item Participation rate (fraction of working-age population in the labor force)
    \item Hours worked by each individual in the labor force
\end{itemize}

A rise in the real hourly wage has a substitution effect (increasing hours worked) and an income effect (reducing hours worked). The four main factors that increase the participation rate are: higher real wages, lower fixed costs of working, lower non-labor income, and changes in tastes favoring work.

\section*{Economic Rent}
Economic rent is the payment to a worker in excess of their reservation wage (the lowest wage they'd accept). Graphically, it's the area above the labor supply curve and below the equilibrium wage.

\section*{Labor Market Equilibrium and Disequilibrium}
Labor market equilibrium occurs when labor demand equals labor supply. Disequilibrium (unemployment) can arise from:
\begin{itemize}
    \item Minimum wages
    \item Trade unions
    \item Scale economies
    \item Insider-outsider distinctions
    \item Efficiency wages (paying above-equilibrium wages to increase productivity)
\end{itemize}

\section*{Minimum Wages and Unemployment}
Minimum wage laws can create unemployment by setting wages above the equilibrium level. The extent depends on the elasticity of labor supply and demand.
