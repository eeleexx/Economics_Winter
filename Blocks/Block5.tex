\chapter*{Block 5: The Firm}

\section*{Economic vs Accounting Definitions of Cost}
\begin{itemize}
    \item \textbf{Accounting Cost:} Explicit costs recorded in the firm’s financial statements, such as wages, rent, and materials.
    \item \textbf{Economic Cost:} Includes both explicit costs and implicit costs (opportunity costs of using resources in their next best alternative).
\end{itemize}
For example, the opportunity cost of an entrepreneur’s time or capital invested in the firm is part of economic cost but not accounting cost.

\section*{Relationship Between Revenue, Cost, and Profit}
Profit (\(\pi\)) is the difference between total revenue (\(TR\)) and total cost (\(TC\)):
\[
\pi = TR - TC,
\]
where:
\[
TR = P \cdot Q, \quad TC = FC + VC.
\]

\section*{The Production Function}
The production function represents the relationship between inputs (e.g., labor \(L\) and capital \(K\)) and output \(Q\):
\[
Q = f(L, K).
\]
\begin{itemize}
    \item \textbf{Short Run:} At least one input (e.g., \(K\)) is fixed.
    \item \textbf{Long Run:} All inputs are variable.
\end{itemize}

\subsection*{Point of Diminishing Marginal Returns}
\begin{itemize}
    \item \textbf{Marginal Product (MP):} The additional output produced by an additional unit of input.
    \[
    MP_L = \frac{\Delta Q}{\Delta L}.
    \]
    \item \textbf{Diminishing Marginal Returns:} Occurs when \(MP_L\) begins to decrease as more units of \(L\) are added while \(K\) remains fixed.
\end{itemize}

\section*{Choice of Production Technique and Input Price}
Firms minimize costs by choosing the optimal combination of inputs based on input prices:
\[
\text{Minimize } C = wL + rK, \quad \text{subject to } Q = f(L, K),
\]
where \(w\) is the wage rate and \(r\) is the rental rate of capital.

\section*{Isoquants and Isocost Curves}
\begin{itemize}
    \item \textbf{Isoquants:} Combinations of \(L\) and \(K\) that produce the same level of output (\(Q_0\)).
    \item \textbf{Isocost Lines:} Combinations of \(L\) and \(K\) that have the same total cost (\(C\)):
    \[
    C = wL + rK \implies K = \frac{C}{r} - \frac{w}{r}L.
    \]
    \item \textbf{Cost Minimization:} The optimal combination of \(L\) and \(K\) occurs where the slope of the isoquant equals the slope of the isocost line:
    \[
    \frac{MPL}{MPK} = \frac{w}{r}.
    \]
\end{itemize}

\section*{Marginal Cost and Marginal Revenue}
\begin{itemize}
    \item \textbf{Marginal Cost (MC):} The additional cost of producing one more unit of output:
    \[
    MC = \frac{\Delta TC}{\Delta Q}.
    \]
    \item \textbf{Marginal Revenue (MR):} The additional revenue from selling one more unit of output:
    \[
    MR = \frac{\Delta TR}{\Delta Q}.
    \]
\end{itemize}

\section*{Profit-Maximizing Level of Output}
The profit-maximizing output (\(Q^*\)) occurs where:
\[
MR = MC.
\]
If the demand curve is \(P = f(Q)\), then:
\[
TR = P \cdot Q, \quad MR = \frac{d(TR)}{dQ}.
\]

\section*{Fixed and Variable Factors in the Short Run}
\begin{itemize}
    \item \textbf{Fixed Factors:} Inputs that do not change with output (e.g., rent, machinery).
    \item \textbf{Variable Factors:} Inputs that change with output (e.g., labor, materials).
\end{itemize}

\section*{Cost Analysis in the Short Run and Long Run}
\subsection*{Short Run:}
\[
TC = FC + VC, \quad ATC = \frac{TC}{Q}, \quad AVC = \frac{VC}{Q}.
\]
\[
MC = \frac{dTC}{dQ}.
\]
The \textbf{law of diminishing returns} causes the \(MC\), \(AVC\), and \(ATC\) curves to be U-shaped.

\subsection*{Long Run:}
\[
LTC = f(Q), \quad LAC = \frac{LTC}{Q}, \quad LMC = \frac{dLTC}{dQ}.
\]
All inputs are variable, and firms can choose the optimal scale of production.

\section*{Returns to Scale and Average Cost Curves}
\begin{itemize}
    \item \textbf{Increasing Returns to Scale:} Doubling inputs more than doubles output (\(LAC\) decreases).
    \item \textbf{Constant Returns to Scale:} Doubling inputs doubles output (\(LAC\) is constant).
    \item \textbf{Decreasing Returns to Scale:} Doubling inputs less than doubles output (\(LAC\) increases).
\end{itemize}

\section*{Output Choice in the Short Run and Long Run}
\begin{itemize}
    \item \textbf{Short Run:} Produce where \(P = MC\), provided \(P \geq AVC\).
    \item \textbf{Long Run:} Produce where \(P = MC = LAC\), ensuring normal profits.
\end{itemize}

\section*{Relationship Between Short-Run and Long-Run Costs}
\begin{itemize}
    \item \textbf{Short-Run Cost Curves:} Based on a fixed level of input (e.g., capital).
    \item \textbf{Long-Run Cost Curves:} Envelop all short-run cost curves, representing the lowest cost for any output level.
\end{itemize}

\section*{Graphical Representation of Cost Curves}
\begin{itemize}
    \item \textbf{Short-Run Curves:} \(MC\) intersects \(AVC\) and \(ATC\) at their minimum points.
    \item \textbf{Long-Run Curves:} \(LMC\) intersects \(LAC\) at its minimum point.
\end{itemize}