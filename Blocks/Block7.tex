\chapter*{Block 7: Pure Monopoly}

\section*{Definition of Pure Monopoly}
A pure monopoly exists when there is only one firm in a market, and it serves as the sole supplier of a good or service with no close substitutes. Additionally, there are high barriers to entry, preventing other firms from entering the market. Examples include utilities like water supply and firms protected by patents.

\section*{Finding the Optimal Price and Output for a Monopolist}
A monopolist maximizes profit by producing the quantity of output where marginal cost (\(MC\)) equals marginal revenue (\(MR\)). Unlike a perfectly competitive firm, a monopolist faces a downward-sloping market demand curve, meaning:
\[
MR < P \quad \text{(price exceeds marginal revenue at all positive output levels)}.
\]

To find the profit-maximizing output:
\begin{enumerate}
    \item Identify the demand curve \(\bigl(P = f(Q)\bigr)\).
    \item Derive the total revenue \(\bigl(TR = P \times Q\bigr)\).
    \item Compute marginal revenue \(\bigl(MR = d(TR)/dQ\bigr)\).
    \item Equate \(MR = MC\) and solve for \(Q^*\) (optimal quantity).
    \item Determine the corresponding price \(P^*\) from the demand curve.
\end{enumerate}

\section*{Relationship Between PED and Monopoly Power}
The monopolist's ability to set a price above marginal cost depends on the price elasticity of demand (\(PED\)). The Lerner Index, a measure of monopoly power, is defined as:
\[
L = \frac{P - MC}{P} = -\frac{1}{PED},
\]
where:
- \(PED < -1\): Elastic demand, limited pricing power.
- \(-1 < PED < 0\): Inelastic demand, greater pricing power.

\section*{Comparison: Monopoly vs Perfect Competition}
Under perfect competition:
\begin{itemize}
    \item Price equals marginal cost (\(P = MC\)), leading to allocative efficiency.
    \item Output (\(Q_c\)) is higher, and price (\(P_c\)) is lower.
\end{itemize}

Under monopoly:
\begin{itemize}
    \item Price exceeds marginal cost (\(P > MC\)), causing allocative inefficiency.
    \item Output (\(Q_m\)) is lower, and price (\(P_m\)) is higher.
    \item Deadweight loss represents the loss in total surplus due to monopoly power.
\end{itemize}

\section*{Price Discrimination and Its Effects}
Price discrimination occurs when a monopolist charges different prices for the same good to different consumers or groups. There are three types:
\begin{enumerate}
    \item \textbf{First-degree (perfect price discrimination):} The monopolist charges each consumer their maximum willingness to pay. Output increases to the level of perfect competition, and the monopolist captures all consumer surplus.
    \item \textbf{Second-degree price discrimination:} Prices vary based on the quantity purchased (e.g., bulk discounts). Consumer surplus is partially captured.
    \item \textbf{Third-degree price discrimination:} Prices differ across identifiable groups with different demand elasticities (e.g., student discounts). Output and profits depend on group elasticities.
\end{enumerate}

\textbf{Effects:}
\begin{itemize}
    \item \textbf{Output:} Generally increases compared to uniform pricing, as more consumers are served.
    \item \textbf{Profits:} Increase as the monopolist captures additional consumer surplus.
\end{itemize}

\section*{Natural Monopoly and Regulation}
A natural monopoly arises when a single firm can produce the market output at a lower cost than multiple firms due to economies of scale. Its long-run average cost (\(LRAC\)) curve slopes downward over the relevant range of output.

Regulatory responses include:
\begin{itemize}
    \item \textbf{Price capping:} Limiting the price to prevent excessive monopoly profits.
    \item \textbf{Marginal cost pricing:} Forcing the monopolist to set \(P = MC\), though this may require subsidies to cover losses.
\end{itemize}

\section*{Social Cost of Monopoly}
Monopolies lead to:
\begin{itemize}
    \item \textbf{Allocative inefficiency:} \(P > MC\), meaning not all mutually beneficial trades occur.
    \item \textbf{Deadweight loss:} The loss of consumer and producer surplus due to reduced output.
\end{itemize}

Despite inefficiency, monopoly profits can incentivize innovation, as firms invest in research and development to maintain or acquire monopoly status.
