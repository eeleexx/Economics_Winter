\chapter*{Block 6: Perfect Competition}

\section*{Definition of Perfect Competition}
Perfect competition is a market structure characterized by the following assumptions:
\begin{itemize}
    \item \textbf{Many firms:} A large number of firms, each negligible in size relative to the entire industry.
    \item \textbf{Homogeneous goods:} All firms produce identical products, making them perfect substitutes.
    \item \textbf{Perfect information:} Buyers and sellers are fully informed about prices and available alternatives.
    \item \textbf{Free entry and exit:} Firms can freely enter or exit the market in the long run.
\end{itemize}
These assumptions lead to the key implication that individual firms are \textbf{price takers}, facing a horizontal demand curve where \(P = AR = MR\).

\section*{Why a Perfectly Competitive Firm Equates Marginal Cost and Price}
For a price-taking firm:
\[
TR = P \cdot Q, \quad AR = \frac{TR}{Q} = P, \quad MR = \frac{d(TR)}{dQ} = P.
\]
The firm maximizes profit (\(\pi\)) by choosing \(Q\) such that:
\[
\pi(Q) = TR(Q) - TC(Q).
\]
Differentiating with respect to \(Q\) gives the first-order condition for profit maximization:
\[
\frac{d\pi}{dQ} = \frac{dTR}{dQ} - \frac{dTC}{dQ} = MR - MC = 0 \implies MR = MC.
\]
Since \(MR = P\) for a competitive firm:
\[
P = MC.
\]

\section*{Profits and Losses: Entry and Exit Dynamics}
In the short run, firms can make profits or losses:
\begin{itemize}
    \item \textbf{Profits} (\(P > ATC\)): New firms enter the market, increasing supply and lowering price until profits are eliminated.
    \item \textbf{Losses} (\(P < ATC\)): Existing firms exit the market, reducing supply and raising price until losses are eliminated.
    \item \textbf{Break-even} (\(P = ATC\)): Firms earn normal profits (zero economic profit), and there is no incentive for entry or exit.
\end{itemize}

In the short run, the firm produces only if \(P \geq AVC\). The shut-down condition is:
\[
P < AVC \implies \text{Firm exits the market.}
\]

\section*{Drawing the Industry Supply Curve}
\subsection*{Short-Run Industry Supply Curve}
The short-run supply curve of an individual firm is its marginal cost (\(MC\)) curve above the average variable cost (\(AVC\)) curve. The industry supply curve is the horizontal summation of all individual firms' supply curves:
\[
Q_s = \sum_{i=1}^N Q_{s_i},
\]
where \(Q_{s_i}\) is the supply of the \(i\)-th firm and \(N\) is the number of firms.

\subsection*{Long-Run Industry Supply Curve}
In the long run:
\begin{itemize}
    \item Firms enter or exit the market, ensuring zero economic profit (\(P = ATC\)).
    \item The long-run supply curve is horizontal if input prices remain constant as the industry expands.
    \item If input prices increase with industry expansion, the long-run supply curve slopes upward.
\end{itemize}

\section*{Comparative Static Analysis of a Competitive Industry}
\subsection*{Effect of a Shift in Market Demand}
An increase in demand shifts the market demand curve outward:
\begin{enumerate}
    \item In the short run, the price rises, increasing firms’ profits.
    \item In the long run, new firms enter, increasing supply and driving the price back to the break-even level (\(P = ATC\)).
\end{enumerate}

\subsection*{Effect of a Change in Costs}
An increase in costs (e.g., higher input prices) shifts the cost curves upward:
\begin{enumerate}
    \item Firms with \(P < AVC\) shut down in the short run, reducing market supply.
    \item In the long run, remaining firms adjust their scale of production, and the market reaches a new equilibrium with higher prices.
\end{enumerate}

\section*{Mathematical Formulas and Conditions}
\subsection*{Revenue and Cost Relationships}
\[
TR = P \cdot Q, \quad AR = \frac{TR}{Q}, \quad MR = \frac{dTR}{dQ}.
\]
\[
TC = FC + VC, \quad ATC = \frac{TC}{Q}, \quad MC = \frac{dTC}{dQ}.
\]

\subsection*{Profit Maximization}
\[
MR = MC, \quad P = MC \text{ (for a competitive firm)}.
\]

\subsection*{Short-Run Supply Decision}
\[
\text{Produce if } P \geq AVC, \quad \text{Shut down if } P < AVC.
\]

\subsection*{Long-Run Equilibrium}
\[
P = MC = ATC.
\]

\section*{Graphical Analysis}
\subsection*{Firm-Level Short-Run Equilibrium}
\[
\text{Profit} = (P - ATC) \cdot Q, \quad \text{Loss} = (ATC - P) \cdot Q.
\]
Graphically, profits and losses are represented by the area between price and the \(ATC\) curve at the equilibrium quantity.

\subsection*{Market Adjustment to Demand Shifts}
Graphs illustrating:
\begin{itemize}
    \item Initial equilibrium (\(P_0, Q_0\)).
    \item Short-run price increase (\(P_1, Q_1\)) due to increased demand.
    \item Long-run price stabilization (\(P_0, Q_2\)) after entry restores equilibrium.
\end{itemize}

\section*{Efficiency in Perfect Competition}
Perfect competition leads to:
\begin{itemize}
    \item \textbf{Allocative efficiency:} \(P = MC\), ensuring resources are distributed optimally.
    \item \textbf{Productive efficiency:} Firms produce at the minimum point of the \(ATC\) curve.
\end{itemize}