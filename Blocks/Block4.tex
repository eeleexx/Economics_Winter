\chapter*{Block 4: Consumer Choice}

\section*{Utility and Consumer Tastes}
Utility is the satisfaction or pleasure that a consumer derives from consuming goods and services. Consumer tastes reflect preferences and are represented by a utility function, \(U(x, y)\), where \(x\) and \(y\) are quantities of two goods.

\section*{Diminishing Marginal Utility}
Marginal utility (\(MU\)) is the additional utility gained from consuming one more unit of a good:
\[
MU_x = \frac{\Delta U}{\Delta x}, \quad MU_y = \frac{\Delta U}{\Delta y}.
\]
\textbf{Diminishing marginal utility} states that as a consumer consumes more of a good, the additional utility derived from each additional unit decreases.

\section*{Diminishing Marginal Rate of Substitution (MRS)}
The Marginal Rate of Substitution (\(MRS\)) is the rate at which a consumer is willing to give up one good for another while maintaining the same level of utility. It is calculated as:
\[
MRS = -\frac{MU_x}{MU_y}.
\]
\textbf{Diminishing MRS} occurs because as the consumer substitutes one good for another, the relative utility of the good being added decreases.

\section*{Indifference Curves Represent Tastes}
Indifference curves represent combinations of two goods that provide the consumer with the same level of utility. Properties of indifference curves:
\begin{itemize}
    \item Downward sloping (trade-offs between goods).
    \item Convex to the origin (reflecting diminishing MRS).
    \item Do not intersect (consistency in preferences).
\end{itemize}

\section*{Budget Line}
The budget line represents all combinations of goods a consumer can afford given income (\(M\)) and prices (\(P_x, P_y\)):
\[
P_x \cdot x + P_y \cdot y = M.
\]
The slope of the budget line reflects the opportunity cost of one good in terms of the other:
\[
\text{Slope} = -\frac{P_x}{P_y}.
\]

\section*{Indifference Curves and Budget Constraints}
A consumer maximizes utility by choosing the point where an indifference curve is tangent to the budget line:
\[
\text{At optimal choice: } MRS = -\frac{P_x}{P_y}.
\]
This means that the rate at which the consumer is willing to substitute one good for another equals the rate at which the market allows substitution.

\section*{Effects of Changes in Income}
An increase in income shifts the budget line outward, allowing the consumer to achieve a higher indifference curve. The change in quantity demanded depends on whether the good is:
\begin{itemize}
    \item \textbf{Normal Good:} Quantity demanded increases with income.
    \item \textbf{Inferior Good:} Quantity demanded decreases with income.
\end{itemize}

\section*{Effects of a Price Change}
A price change pivots the budget line. The impact on quantity demanded can be decomposed into:
\begin{itemize}
    \item \textbf{Substitution Effect:} Change due to relative price change, holding utility constant.
    \item \textbf{Income Effect:} Change due to a change in purchasing power.
\end{itemize}

\section*{Income and Substitution Effects}
For a normal good, both effects lead to an increase in quantity demanded when the price falls. For an inferior good, the income effect may partially offset the substitution effect.

\section*{Market Demand Curve from Individual Demand Curves}
The market demand curve is the horizontal summation of individual demand curves:
\[
Q^D_{\text{market}} = \sum_{i=1}^n Q^D_i.
\]